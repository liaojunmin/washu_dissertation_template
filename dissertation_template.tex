%% LyX 2.1.2 created this file.  For more info, see http://www.lyx.org/.
%% Do not edit unless you really know what you are doing.
\documentclass[12pt,oneside,english]{book}
\usepackage[T1]{fontenc}
\usepackage[latin9]{inputenc}
\usepackage[letterpaper]{geometry}
\geometry{verbose,tmargin=1in,bmargin=1.43in,lmargin=1in,rmargin=1in}
\pagestyle{plain}
\setcounter{secnumdepth}{3}
\usepackage{babel}
\usepackage{url}
\usepackage{amsthm}
\usepackage{amsmath}
\usepackage{amssymb}
\usepackage{graphicx}
\usepackage{setspace}
\doublespacing
\usepackage[unicode=true,pdfusetitle,
 bookmarks=true,bookmarksnumbered=true,bookmarksopen=false,
 breaklinks=true,pdfborder={0 0 0},backref=false,colorlinks=false]
 {hyperref}

\makeatletter

%%%%%%%%%%%%%%%%%%%%%%%%%%%%%% LyX specific LaTeX commands.
%% Because html converters don't know tabularnewline
\providecommand{\tabularnewline}{\\}

\@ifundefined{date}{}{\date{}}
%%%%%%%%%%%%%%%%%%%%%%%%%%%%%% User specified LaTeX commands.
\makeatletter
\renewcommand*\l@chapter{\@dottedtocline{1}{0em}{1.5em}}
\makeatother

\makeatother

\begin{document}
\thispagestyle{empty}

\begin{spacing}{1.15}
\noindent \begin{center}
WASHINGTON UNIVERSITY IN ST. LOUIS\\
Division of Biology and Biomedical Sciences Neurosciences
\par\end{center}

\noindent \begin{center}
\vspace*{\fill}

\par\end{center}

\noindent \begin{center}
Dissertation Examination Committee:\\
Katherine Davidsen, Chair \\
Michael Randolf, Co-Chair \\
Richard Lewis \\
Hillary O\'Connell \\
Jack Taylor
\par\end{center}

\noindent \begin{center}
\vspace*{\fill}

\par\end{center}

\noindent \begin{center}
A Mock Thesis on the Proper Formatting of Dissertations and Theses
for Arts \& Sciences Graduate Students\\
by\\
Paige Turner
\par\end{center}

\noindent \begin{center}
\vspace*{\fill}

\par\end{center}

\noindent \begin{center}
A dissertation presented to the \\
Graduate School of Arts and Sciences\\
of Washington University in \\
partial fulfillment of the \\
requirements for the degree\\
of Doctor of Philosophy
\par\end{center}

\noindent \begin{center}
\vspace*{\fill}

\par\end{center}

\noindent \begin{center}
May 2015\\
St. Louis, Missouri
\par\end{center}
\end{spacing}

\clearpage{}

\pagenumbering{roman}

\thispagestyle{empty}

\vspace*{\fill}


\begin{center}
\copyright ~2015, Paige Turner
\par\end{center}

\vspace*{\fill}


\clearpage{}

\begin{spacing}{1.15}

\renewcommand*{\contentsname}{Table of Contents}

\begin{spacing}{1.15}
\tableofcontents{}
\end{spacing}

\clearpage{}

\phantomsection

\addcontentsline{toc}{chapter}{\numberline{}List of Figures}

\begin{spacing}{1.15}
\listoffigures

\end{spacing}

\clearpage{}

\phantomsection

\addcontentsline{toc}{chapter}{\numberline{}List of Tables}

\begin{spacing}{1.15}
\listoftables

\end{spacing}

\end{spacing}

\clearpage{}

\phantomsection

\addcontentsline{toc}{chapter}{\numberline{}Acknowledgements}


\chapter*{Acknowledgments}

An acknowledgments page must be included in your final dissertation
or thesis. If you wish to include a special dedication you can either
use it to close the acknowledgments page or place it on the page that
immediately follows. The acknowledgments page should be listed in
the table of contents. Place it after the final list used in the document,
and before any dedication, abstract, or epigraph that is included.

It is appropriate to acknowledge sources of academic and financial
support; some fellowships and grants require acknowledgment. 

We offer special thanks to the Washington University School of Engineering
for allowing us to use their dissertation and thesis template as a
starting point for the development of this document.

\begin{flushright}
Paige Turner
\par\end{flushright}

\noindent Washington University in St. Louis

\noindent May 2015

\clearpage{}

\vspace*{\fill}


\begin{center}
Dedicated to my family.%
\footnote{If you include a special dedication as shown here be sure to keep
it brief and center it on the page both horizontally and vertically.
Alternatively, you may remove this page altogether, and a special
dedication can be placed as the final paragraph of your acknowledgments
page. Do not include the dedication page in your table of contents.%
}
\par\end{center}

\vspace*{\fill}


\clearpage{}

\phantomsection

\addcontentsline{toc}{chapter}{\numberline{}Abstract}

\noindent \begin{center}
ABSTRACT OF THE DISSERTATION 
\par\end{center}

\noindent \begin{center}
A Mock Thesis on the Proper Formatting of Dissertations and Theses\\
for Arts \& Sciences Graduate Students\\
by\\
Paige Turner 
\par\end{center}

\noindent \begin{center}
Doctor of Philosophy in Biology and Biomedical Sciences Neurosciences\\
Washington University in St. Louis, 2015\\
Professor Katherine Davidsen, Chair\\
Professor Michael Randolf, Co-Chair
\par\end{center}

After removing these comments, begin typing the body of your abstract
here, double-spaced. Your font should be 12-point (which is the text
of this sample paragraph). No part of the abstract should be bolded.
If this is for your master\textquoteright s degree, be sure to change
all occurrences of the word \textquotedblleft dissertation\textquotedblright{}
to display as \textquotedblleft thesis,\textquotedblright{} and change
\textquotedblleft Doctor of Philosophy\textquotedblright{} or \textquotedblleft Doctor
of Liberal Arts\textquotedblright{} to \textquotedblleft Master of
Arts,\textquotedblright{} \textquotedblleft Master of Liberal Arts,\textquotedblright{}
or \textquotedblleft Master of Fine Arts,\textquotedblright{} whichever
applies. In the abstract heading above, make sure you use the year
your degree is to be officially earned. Be sure to use your full name
as it is recorded in WebSTAC, your dissertation or thesis advisor\textquoteright s
full name(s) wherever appropriate, and the correct title of your degree
whenever referencing it. The title of your degree will not always
be the same as the title of your department or program, so please
check with your departmental administrative assistant and advisor(s)
to be sure you are using the correct degree title. Please note that
an abstract is required for all dissertation submissions in ProQuest.
An abstract is optional for master\textquoteright s thesis submissions.

\clearpage{}

\pagenumbering{arabic}


\chapter{Parts of the Dissertation}

This chapter describes the components of a dissertation or thesis.
You may not have to include all components described here, but you
must follow the prescribed order for the components you do include.
Table \ref{tab:The-following-items} lists the required and optional
components in the order that they should appear. Your manuscript should
include three main parts: the front matter, the body, and the back
matter. Each of these parts is described below.

\begin{table}[h]
\protect\caption{Required and Optional Components\label{tab:The-following-items}}


\begin{centering}
{\scriptsize{}}%
\begin{tabular}{ccccc}
\hline 
{\scriptsize{}Major Part} & {\scriptsize{}Thesis Component} & {\scriptsize{}Required} & {\scriptsize{}Optional} & {\scriptsize{}Page Numbering}\tabularnewline
\hline 
{\scriptsize{}Front Matter} & {\scriptsize{}Title page} & {\scriptsize{}$\checkmark$} &  & {\scriptsize{}counted, not numbered}\tabularnewline
 & {\scriptsize{}Copyright page} &  & {\scriptsize{}$\checkmark$} & {\scriptsize{}neither counted, nor numbered}\tabularnewline
 & {\scriptsize{}Table of Contents} & {\scriptsize{}$\checkmark$} &  & {\scriptsize{}begins on a page numbered ii}\tabularnewline
 & {\scriptsize{}List of Figures} &  & {\scriptsize{}$\checkmark$} & {\scriptsize{}{[}lowercase Roman numerals continue{]}}\tabularnewline
 & {\scriptsize{}List of Illustrations} &  & {\scriptsize{}$\checkmark$} & {\scriptsize{}{[}lowercase Roman numerals continue{]}}\tabularnewline
 & {\scriptsize{}List of Tables} &  & {\scriptsize{}$\checkmark$} & {\scriptsize{}{[}lowercase Roman numerals continue{]}}\tabularnewline
 & {\scriptsize{}List of Abbreviations} &  & {\scriptsize{}$\checkmark$} & {\scriptsize{}{[}lowercase Roman numerals continue{]}}\tabularnewline
 & {\scriptsize{}Acknowledgments} & {\scriptsize{}$\checkmark$} &  & {\scriptsize{}{[}lowercase Roman numerals continue{]}}\tabularnewline
 & {\scriptsize{}Dedication} &  & {\scriptsize{}$\checkmark$} & {\scriptsize{}{[}lowercase Roman numerals continue{]}}\tabularnewline
 & {\scriptsize{}Abstract page} &  & {\scriptsize{}$\checkmark$} & {\scriptsize{}{[}lowercase Roman numerals continue{]}}\tabularnewline
 & {\scriptsize{}Preface} &  & {\scriptsize{}$\checkmark$} & {\scriptsize{}{[}lowercase Roman numerals continue{]}}\tabularnewline
{\scriptsize{}Body} & {\scriptsize{}Epigraph} &  & {\scriptsize{}$\checkmark$} & {\scriptsize{}begins on a page numbered 1}\tabularnewline
 & {\scriptsize{}Chapters} & {\scriptsize{}$\checkmark$} &  & {\scriptsize{}{[}Arabic numerals begin or continue{]}}\tabularnewline
{\scriptsize{}Back Matter} & {\scriptsize{}References} & {\scriptsize{}$\checkmark$} &  & {\scriptsize{}{[}Arabic numerals continue{]}}\tabularnewline
 & {\scriptsize{}Appendices} &  & {\scriptsize{}$\checkmark$} & {\scriptsize{}{[}Arabic numerals continue{]}}\tabularnewline
 & {\scriptsize{}Curriculum Vitae} &  & {\scriptsize{}$\checkmark$} & {\scriptsize{}{[}Arabic numerals continue{]}}\tabularnewline
\hline 
\end{tabular}
\par\end{centering}{\scriptsize \par}

{\scriptsize{}The above items may be included in your dissertation
or thesis, in the order in which they are listed. Any optional components,
if used, must be included in the table of contents, unless noted below.
Do not include Dedication and Epigraph in the table of contents. There
are two options for the placement of references; they can be listed
at the end of each chapter, or at the end of the document. Do not
put your Social Security Number, birthdate, or birthplace on your
C.V.}
\end{table}



\section{Front Matter}

The front matter includes all material that appears before the beginning
of the body of the text. Number all front matter pages (except the
title page and the optional copyright page) with lowercase roman numerals,
starting with ii, centered just above the bottom margin. Each of the
following sections should begin on a new page.


\subsection{Title Page}

Format the title page so that it is centered vertically and horizontally
on the page with equal amounts of white space from top and bottom
margins. Include a 1 inch margin on all sides. Use a 12-point regular
font. If you are writing a thesis, substitute the word \textquotedblleft thesis\textquotedblright{}
wherever the word \textquotedblleft dissertation\textquotedblright{}
appears in this document; do not include your committee members. The
date on the title page should reflect the month and year the degree
is to be officially earned, and should be one of the following months:
December, May, or August. Do not include a number on the title page.
See Appendix for further details. In most cases your dissertation
title should be in \textquotedblleft Title Case\textquotedblright{}
unless a specific format is required by your discipline. Be certain
to use your own full name (as recorded in WebSTAC). 


\subsection{Copyright Page }

It is always suggested that upon completion of the text, the student
add the copyright symbol \copyright ~ with the year and the student\textquoteright s
name on one line, on a page following the title page. Format your
copyright page exactly as it is shown in this template. 

Example:

\noindent \begin{center}
\copyright ~ 2015, Paige Turner
\par\end{center}

Once you create a work, it is automatically protected by U.S. copyright
law with you as the author. You do not need to register the copyright
with the U.S. Copyright Office, though doing so provides certain advantages.
More information about copyright registration can be found at \url{http://libguides.wustl.edu/copyright/registration}. 


\subsection{Table of Contents }

The words \textquotedblleft Table of Contents\textquotedblright{}
must appear in chapter title style at the top of the page. It must
include the page numbers of all front and back matter elements, unless
otherwise specified. The table of contents must include the page numbers
of all chapters and sections of your dissertation or thesis. In addition,
it may include the page numbers of all subsections. Chapter titles
may be typed in plain or bold font. All titles and headings must be
followed by a page number. 

Make certain that any long titles align nicely with the body of text.
Multi-lined chapter titles or section titles should break at a logical
point and align in a manner allowing the titles to be read clearly,
without confusion. Sometimes this will mean forcing a line break at
a logical point and relies on your own good judgment. 


\subsection{List of Figures }

If one or more figures are used in the document, there must be a list
of all figures. The list should be spaced at 1.15. Begin each listing
on a new line. Format the list of figures the same way the table of
contents is formatted, but put the words \textquotedblleft List of
Figures\textquotedblright{} in the heading. 


\subsection{List of Tables }

If one or more tables are used in the document, there must be a list
of all tables. The list should be spaced at 1.15. Begin each listing
on a new line. Format the list of tables the same way the table of
contents is formatted, but put the words \textquotedblleft List of
Tables\textquotedblright{} in the heading. 


\subsection{List of Abbreviations }

Include a list of abbreviations only if you use abbreviations that
are not common in your field. Arrange the list alphabetically. Type
the words \textquotedblleft List of Abbreviations\textquotedblright{}
in chapter title style at the top of your list. 


\subsection{Acknowledgments }

An acknowledgments section must be included. Use it to thank those
who supported your research through contributions of time, money,
or other resources. Some grants require an acknowledgment. Type the
word \textquotedblleft Acknowledgments\textquotedblright{} in the
chapter title style at the top of your page. If the acknowledgments
fill more than one page, put the heading only on the first page. 


\subsection{Dedication }

The dedication page is optional. If you decide to include a separate
dedication page, make it short and center it on the page, both horizontally
and vertically. Do not include it in your table of contents. See page
vii for more detailed information. 


\subsection{Abstract}

Page An abstract page is optional in the dissertation or thesis, but
will be required when you submit your manuscript electronically. Format
the abstract page precisely as shown in the front matter of this document.
See Appendix for further details. 


\subsection{Preface }

A preface is optional. If you include a preface, use it to explain
the motivation behind your work. Format the preface the same way the
acknowledgments section is formatted, but use the word \textquotedblleft Preface\textquotedblright{}
in the heading. 


\section{Body of the Dissertation or Thesis }

The body of the dissertation or thesis should be divided into chapters,
sections, and subsections as required by your discipline, and should
be numbered as in this template. Divisions smaller than subsections
may be used, but they should not be labeled with numbers (see 2.5.1
for more information). 


\section{Back Matter }

The back matter includes all material that appears after the body
of the text. 


\subsection{References/Bibliography/Works Cited }

There are two options for the placement of references; they can be
listed at the end of each chapter or at the end of the document. What
you call this section and how you format it should follow the usual
convention of your discipline and be acceptable to your committee.
Depending on how you title and where you place this section, type
the appropriate words in either the section heading or chapter title
format at the top of a new page. Single space your citations and skip
a line between each one. Regardless of where you choose to place your
references, they must be listed in the table of contents. If placed
at the end of your document, this section should follow the conclusion
of the text. 


\subsection{Appendices }

Appendices may be used for including reference material that is too
lengthy or inappropriate for the dissertation or thesis body. If one
appendix is included, an appendix title is optional. If more than
one appendix is included, each one should be titled and lettered.
In general, appendices should be formatted like chapters. However,
they may be single-spaced and/or include photocopied or scanned materials.
If these are used, you must add page numbers at the bottom, putting
those page numbers in square brackets to indicate that they are not
part of the original document. 


\subsection{Curriculum Vitae }

Including a Curriculum Vitae (C.V.) with your dissertation or thesis
is optional. If you choose to include your C.V., it should include
your name, the month \& year you will be earning your degree, and
relevant academic and professional achievements. It may also include
your publications and professional society memberships. If included,
your vita should be the last page(s) of your document. Note that personally
identifiable information such as birth date, place of birth, and social
security number should NOT be included.

\clearpage{}


\chapter{Dissertation or Thesis Format }

The following guidelines offer you some degree of flexibility in formatting
your thesis or dissertation. Whichever options you choose to use,
you must use them consistently throughout the document. 


\section{Margins }

Your printed output must reflect physically measurable top, bottom,
left, and right margins of 1 inch. Some systems\textquoteright{} settings
produce varying results when printing to different printers, so be
sure to measure your output. Remember, nothing, not even page numbers,
should print in the margins. 


\section{Page Numbers }

All pages numbers should be placed on the center of each page immediately
above the bottom margin. Number all pages in your document except
for the title page and the optional copyright page which might follow
the title page. Number the \textquotedblleft front matter\textquotedblright{}
pages (i.e., the pages that come prior to the main body of text, prior
to chapter 1) with lowercase roman numerals, starting with ii (remember,
the title page is counted but not numbered, and the copyright page
is neither counted nor numbered). The body of the dissertation or
thesis should be numbered with Arabic numerals starting with 1 and
should begin on the epigraph page (optional), the introduction, if
one is used, or on the first page of the first chapter. Options are
summarized in Table 1.1 on page two.


\section{Text }

This template uses a 12-point, Times New Roman font throughout and
is recommended for your dissertation or thesis. However, should you
choose to use a different font, you should match the font size as
close as possible. Use double-spacing for body text. Use either left
justification with a ragged right edge or full justification. 


\section{Chapter Titles }

Begin each chapter on a new page. You may start the chapter title
below the top margin or you may leave some space and start the chapter
title up to 3 inches from the top edge of the page. The font size
for chapter titles should be no larger than 24. There are two options
for formatting the chapter title: 
\begin{enumerate}
\item Type the word \textquotedblleft Chapter\textquotedblright{} followed
by the chapter number, skip a line, and type the chapter title on
the following line; or 
\item Type the chapter number followed by the chapter title, all on the
same line.
\end{enumerate}

\section{Section Headings \& Numbering }

Headings may be typed above or on the same line as the sections they
label. Type the chapter number and section number before the section
title. The font size for section headings should be no larger than
18. 


\subsection{Subsection Headings \& Numbering }

This should follow the usual convention of your discipline and be
acceptable to your committee; if your discipline calls for them they
must be in included in the table of contents. Type the chapter number,
section number and subsection number before the subsection title.
The font size for subsection headings should be no larger than 14.


\subsection{Headings for Divisions Smaller than Subsections }

Do not number headings for divisions smaller than subsections. These
are not included in the table of contents. Headings may be typed above
or on the same line as the sections they label. Divisions smaller
than subsections should be the same font size as the body text. 


\section{Figures and Tables }

There are two options for numbering your figures and tables; choose
one and be consistent in its use throughout your document. In either
case, the name and description of the figure or table should be single
spaced and can be a smaller font size. 
\begin{enumerate}
\item Maintain a separate numbering sequence for your list of figures and
list of tables. Label figures with the word \textquotedblleft Figure\textquotedblright{}
and tables with the word \textquotedblleft Table\textquotedblright ;
or 
\item Label both figures and tables with the word \textquotedblleft Figure\textquotedblright{}
and maintain one numbering sequence. 
\end{enumerate}
Place figures and tables as close to their reference in the text as
possible. Do not let figures or tables spill out into the margins. 
\begin{enumerate}
\item Place the figure number and title below each figure (or table labeled
as a figure). 
\item Place the table number and title above each table labeled as a table. 
\end{enumerate}
\begin{figure}[h]
\includegraphics[width=0.5\textwidth]{test_img}

\protect\caption{You can left justify your figure.}


\end{figure}


\begin{figure}[h]
\begin{centering}
\includegraphics[width=0.5\textwidth]{test_img}
\par\end{centering}

\protect\caption{You can center your figure.}
\end{figure}



\section{Lists }

You may include lettered, numbered, or bulleted lists in your document.
Use consistent punctuation and capitalization throughout each list.
Lists may be indented. 


\section{Footnotes and Endnotes }

You may use footnotes or endnotes for brief notes that are not appropriate
for the body of the text.%
\footnote{The use of endnotes and footnotes should follow the usual convention
of your discipline and be acceptable to your committee.%
} Use either footnotes or endnotes consistently throughout your dissertation
or thesis. Endnotes are positioned at the end of each chapter. Single-space
within each footnote or endnote. Footnotes should be numbered consecutively
within a chapter; numbers should restart with each new chapter. Endnotes
should be numbered consecutively through the entire body of your dissertation
or thesis. 


\section{Quotations }

You must use quotation marks and parenthetical references to indicate
words that are not your own. Put quotation marks around short quotes.
Put long quotes in separate single-spaced paragraphs, indented up
to 1 inch from the left margin (these are called block quotations).
Kate Turabian, editor of official publications and dissertation secretary
at the University of Chicago for over 25 years, distinguishes short
and long quotes as follows: 
\begin{quote}
\begin{singlespace}
Short, direct prose quotations should be incorporated into the text
of the paper and enclosed in double quotation marks: \textquotedblleft One
small step for man; one giant leap for mankind.\textquotedblright{}
But in general a prose quotation of two or more sentences which at
the same time runs to four or more lines of text in a paper should
be set off from the text and indented in its entirety\dots {[}8{]} \end{singlespace}

\end{quote}

\section{Equations }

Equation numbering and formatting should follow the usual convention
of your discipline and be acceptable to your committee. 

Equations may be set in-line with the text or numbered and placed
in separate paragraphs. Use the same numbering style for equations
as you would for figures and tables. Here is an example of an equation
set in-line with a paragraph: $E=mc^{2}$. Here is an example equation
placed in a separate paragraph: 
\begin{equation}
E=mc^{2}
\end{equation}


\clearpage{}

\phantomsection

\addcontentsline{toc}{chapter}{\numberline{}Bibliography}

\bibliographystyle{ieeetr}
\nocite{*}
\bibliography{test}


\clearpage{}

\phantomsection

\addcontentsline{toc}{chapter}{\numberline{}Appendix}

\appendix

\chapter*{Appendix }


\section*{Title page }

The second line (or second and third lines) on the page must name
your administrative unit.
\begin{itemize}
\item If your degree is offered by one department of Arts \& Sciences on
the Danforth Campus, your unit is that department: 
\end{itemize}
\begin{center}
Department of East Asian Languages \& Cultures
\par\end{center}
\begin{itemize}
\item For a co-sponsored degree such as English \& Comparative Literature,
credit both: 
\end{itemize}
\noindent \begin{center}
Department of English 
\par\end{center}

\noindent \begin{center}
Program in Comparative Literature
\par\end{center}
\begin{itemize}
\item For the Division of Biology \& Biomedical Sciences, credit DBBS and
your program: 
\end{itemize}
\noindent \begin{center}
Division of Biology \& Biomedical Sciences Neurosciences
\par\end{center}
\begin{itemize}
\item Credit only the program for any of the non-DBBS PhDs on the Medical
Campus: 
\end{itemize}
\noindent \begin{center}
Program in Movement Science \\
or \\
Program in Speech \& Hearing
\par\end{center}
\begin{itemize}
\item If you are in a department in Engineering, credit the School and the
department: 
\end{itemize}
\noindent \begin{center}
School of Engineering \& Applied Science 
\par\end{center}

\noindent \begin{center}
Department of Biomedical Engineering
\par\end{center}
\begin{itemize}
\item If you are in social work or business, your administrative unit is
the School: 
\end{itemize}
\noindent \begin{center}
Brown School of Social Work\\
 or \\
Olin Business School
\par\end{center}


\section*{Abstract page }

Frequent confusion occurs because your abstract heading names your
degree rather than your administrative unit, so it may \textendash{}
or may not \textendash{} match your title page in that respect.
\begin{itemize}
\item For the Division of Biology \& Biomedical Sciences: 
\end{itemize}
\noindent \begin{center}
Doctor of Philosophy in Biology and Biomedical Sciences Neurosciences
\par\end{center}
\begin{itemize}
\item For a co-sponsored degree such as English \& Comparative Literature,
credit both: 
\end{itemize}
\noindent \begin{center}
Doctor of Philosophy in English and Comparative Literature
\par\end{center}
\begin{itemize}
\item If your degree is offered by one department of Arts \& Sciences on
the Danforth Campus: 
\end{itemize}
\noindent \begin{center}
Doctor of Philosophy in Chemistry
\par\end{center}
\end{document}
